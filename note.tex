\documentclass{article}
\usepackage[utf8]{inputenc}
\usepackage{amsmath}
\usepackage{url}
\usepackage{amssymb}
\DeclareMathOperator{\R}{\mathbb{R}}
\DeclareMathOperator{\diag}{diag}
\title{lp interior point}
\author{zhaof17 }
\date{April 2021}

\begin{document}

\maketitle

\section{Introduction}
We use barrier method to solve LP in standard form:
\begin{align}\label{eq:dual}
    \max \,& b^T y \\
    s.t. \,& A^T y \leq c 
\end{align}
where $b,y \in \R^{m}, c\in \R^{n} $ and $A$
is an $m\times n$ matrix.

Writing $A=[a_1, \dots, a_n]$ and Consider the equivalent problem:
\begin{equation}\label{eq:tby}
    \max t b^T y + \sum_{s=1}^{n}\log (c_s- a_s^T y)
\end{equation}
for $t>0$.
For given $y$ and $t$, using Newton's method to solve
\eqref{eq:tby}, $\Delta y$ is given by:
\begin{equation}\label{eq:leq}
    \sum_{s=1}^n \frac{a_sa_s^T}{(a_s^T y - c_s)^2} \cdot \Delta y = F(y)
\end{equation}
where $F(y):= tb+\sum_{s=1}^n \frac{a_s}{a_s^T y - c_s}$
is a function from $\R^{m}$ to $\R^{m}$.
The coefficients of the linear equation system in \eqref{eq:leq} can be written in concise form as: $F(y) = tb + A d$ where $d = \frac{1}{A^T y - c} \in \R^{n}$.
Further, the negative Jacobian is:
\begin{equation}
    -J_F(y) = A \diag(d^2) A^T
\end{equation}
\section{Primal dual formulation}
Consider the canonical form of LP:
\begin{align}
    \min \,& c^T x \\
    s.t. \,& A^T x = b, x\geq 0 
\end{align}
Its duality is given by
\eqref{eq:dual}.
If we solve
\begin{align}
    \min \,& c^T x - \sigma \sum_{i=1}^n\log(x_i) \\
    s.t. \,& A^T x = b
\end{align}
for $\sigma>0$.
The optimal condition is given by
$c= A^T y +  \sigma (\frac{1}{x_1}, \dots, \frac{1}{x_n})^T$ and $Ax=b$.
Let $s_i = \frac{\sigma}{ x_i}$. Then we have
$A^T y + s = c$ and $SX=\sigma$.
where $S=\diag(s_1, \dots, s_n), X=\diag(x_1,\dots, x_n)$. The joint equation
is $F_t(x,y,s)=0$ where
\begin{align}\label{eq:F}
    F(x,y,s)=\begin{pmatrix}
    A^Ty + s - c \\
    Ax - b \\
    XSe - \sigma
    \end{pmatrix}
\end{align}
Using Newton's method to solve
\eqref{eq:F} gives the following update scheme:
\begin{equation}\label{eq:update_F}
    \begin{pmatrix}
    0 & A^T & I_n \\
    A & 0 & 0\\
    S & 0 & X
    \end{pmatrix}
    \begin{pmatrix}
    \Delta x\\ \Delta y \\ \Delta s
    \end{pmatrix}
    = F(x,y,s)
\end{equation}
We reformulate \eqref{eq:dual} as 
\begin{align}\label{eq:dual2}
    \max \,& b^T y \\
    s.t. \,& A^T y + s = c , s\geq 0
\end{align}
Suppose the solution triple to $F_t(x,y,s)=0$ 
is given by $(x_t^*, y_t^*, s_t^*)$.
Then $(y_t^*,s_t^*)$ is feasible to \eqref{eq:dual2}. The dual value at
$(y_t^*,s_t^*)$ is given by
$b^T y_t^* = b^T y_t^* - (A^T y_t^* + s_t^* - c)^Tx_t^* = c^T x_t^* - \sum_{i=1}^n (x^*_t)_i(s^*_t)_i = c^T x_t^* -\frac{n}{t}$.
That is, the dual gap is $\frac{n}{t}$,
and $\sigma$ represents the average dual gap.
Below we write the primal dual interior point
method for LP.
\begin{enumerate}
    \item Starting from $x^{(0)}>0, s^{(0)}>0, y^{(0)}$
    for given decaying coefficient $\mu$.
    (arbitrary). Let $t=0$.
    \item Determine the new average dual gap
    $\sigma = \mu\frac{1}{n}\sum_{i=1}^n x^{(t)}_i s^{(t)}_i$
    \item If $\sigma$ and $F(x,y,s)$ are small
    enough, terminate;
    \item Solve \eqref{eq:update_F} to get the update direction
    $\Delta x, \Delta y, \Delta s$.
    \item Update the new value
    $x^{(t+1)} = x^{(t)} + v\Delta x $
    $y^{(t+1)} = y^{(t)} + v\Delta y $
    $s^{(t+1)} = s^{(t)} + v\Delta s $
    where $v$ is chosen such that
    $x^{(t+1)}>0,s^{(t+1)}>0$ are guaranteed.
    \item Let $t\leftarrow t+1$. Return to step 2.
\end{enumerate}

\begin{thebibliography}{9}
\bibitem{ch10} Interior-Point Methods for Linear Programming,
\url{https://people.inf.ethz.ch/fukudak/lect/opt2011/aopt11note4.pdf}
\bibitem{stephen} Convex optimization, Stephen Boyd
\end{thebibliography}
\end{document}
